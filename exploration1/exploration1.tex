\documentclass{article}
\usepackage{times,amssymb,amsmath}
\input{../cs480preamble.tex}

\newcommand{\term}[1]{\textsc{#1}}

\begin{document}

\headerbox{Exploration 1: Or why we should get an A for failing}{Riley Crebs, Devin Doman, Joe Dzado, Brandon Lyons, Graham Tibbitts}

Riemann

\section{Algebraic Screw-up}
What follows is a semi-faithful recounting of a failed attempt to solve the problem algebraically.
\subsection{Initial assumptions}
These are some assumptions we made in our first attempt to solve the problem with algebra:
1. $f(2n) = f(n)$\\
2. $f(2n-1) = f(n) + 1$\\

(1) turns out to be true, because of the fractal nature of $f(n)$.\\
(2) is false, although assuming it to be true makes for some interesting algebra.\\

\subsection{Flawed reasoning}
$$\sum_{n=1}^\infty \left( \frac{f(n)}{n(n-1)} \right)=\sum_{n=1}^\infty \left( \frac{f(n)}{n} \right)-\sum_{n=1}^\infty \left( \frac{f(n)}{n-1} \right)$$\\
$$\sum_{n=1}^\infty \left( \frac{f(n)}{n} \right)-\sum_{n=1}^\infty \left( \frac{f(n)}{n-1} \right)$$

\end{document}