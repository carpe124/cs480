\documentclass{article}
\usepackage{times,amssymb,amsmath}
\input{../cs480preamble.tex}

\newcommand{\term}[1]{\textsc{#1}}

\begin{document}

\headerbox{Exploration 1: Or why we should get an A for failing}{Riley Crebs, Devin Doman, Joe Dzado, Brandon Lyons, Graham Tibbitts}

Riemann

\section{Algebraic Screw-up - by Graham}
What follows is a semi-faithful recounting of a failed attempt to solve the problem algebraically.
\subsection{Initial assumptions}
These are some assumptions we made in our first attempt to solve the problem with algebra:
\begin{enumerate}
    \item $f(2n) = f(n)$
    \item $f(2n-1) = f(n) + 1$
\end{enumerate}

(1) turns out to be true, because of the fractal nature of $f(n)$.
(2) is false, although assuming it to be true makes for some interesting algebra, as shown in the following\\

\subsection{Irrelevant tangent}
$$\sum_{n=1}^\infty \left( \frac{f(n)}{n(n-1)} \right)=\sum_{n=1}^\infty \left( \frac{f(n)}{n} \right)-\sum_{n=1}^\infty \left( \frac{f(n)}{n-1} \right)$$\\
$$=\sum_{n=1}^\infty \left( \frac{f(n)}{n} \right)-\sum_{n=1}^\infty \left( \frac{f(n)}{n+1} \right)$$\\
Split summation into odds and evens
$$=\sum_{n=1}^\infty \left( \frac{f(2n)}{2n} \right)+\sum_{n=1}^\infty \left( \frac{f(2n-1)}{2n-1} \right)
-\sum_{n=1}^\infty \left( \frac{f(2n)}{2n+1} \right)-\sum_{n=1}^\infty \left( \frac{f(2n-1)}{2n} \right)$$\\
$$=\sum_{n=1}^\infty \left( \frac{f(n)}{2n} \right)+\sum_{n=1}^\infty \left( \frac{f(n)+1}{2n-1} \right)
-\sum_{n=1}^\infty \left( \frac{f(n)}{2n+1} \right)-\sum_{n=1}^\infty \left( \frac{f(n)+1}{2n} \right)$$\\
$$=\sum_{n=1}^\infty \left( \frac{f(n)-(f(n)+1)}{2n} \right)
+\sum_{n=1}^\infty \left( \frac{(f(n)+1)(2n+1)-(f(n)(2n-1)}{(2n-1)(2n+1)} \right)$$\\
$$=\sum_{n=1}^\infty \left( \frac{-1}{2n} \right)
+\sum_{n=1}^\infty \left( \frac{2nf(n)+f(n)+2n+1-2nf(n)-f(n)}{(2n-1)(2n+1)} \right)$$\\
$$=\sum_{n=1}^\infty \left( \frac{-1}{2n} \right)
+\sum_{n=1}^\infty \left( \frac{2n+1}{(2n-1)(2n+1)} \right)$$\\
$$=\sum_{n=1}^\infty \left( \frac{-1}{2n} \right)
+\sum_{n=1}^\infty \left( \frac{1}{2n-1} \right)$$\\
$$=\sum_{n=1}^\infty \left( \frac{1}{2n-1} \right)
-\sum_{n=1}^\infty \left( \frac{1}{2n} \right)$$\\
$$=\frac{1}{1} - \frac{1}{2} + \frac{1}{3} - \frac{1}{4} \cdots = ln2$$\\
Reaching this result was exciting because it was so close to the desired result ($2ln2$). However, since it was based on a flawed assumption ($f(2n-1) \neq f(n) + 1$).

\subsection{Better assumptions}
After we realized that assumption (2) is false, we tried to come up with a similar assumption that was demonstrably true. Here is what we came up with:\\

3. $f(2n + 1)=f(n) + 1$

\subsubsection{Proof}

The popcount of an odd number is always one more than the popcount of the previous number. I.e.,
$$f(2n + 1)=f(2n)+1$$
$$f(2n + 1)=f(n)+1$$

\subsection{Another dead end}

Using (3), we can go through the same type of logic we did before. Unfortunately, that doesn't prove to be as interesting mainly because we have to change the indicies of one of the summations:
$$=\sum_{n=1}^\infty \left( \frac{f(2n)}{2n} \right)+\sum_{n=0}^\infty \left( \frac{f(2n+1)}{2n+1} \right)
-\sum_{n=1}^\infty \left( \frac{f(2n)}{2n+1} \right)-\sum_{n=0}^\infty \left( \frac{f(2n+1)}{2n+2} \right)$$\\
$$=\sum_{n=1}^\infty \left( \frac{f(n)}{2n} \right)+\sum_{n=0}^\infty \left( \frac{f(n)+1}{2n+1} \right)
-\sum_{n=1}^\infty \left( \frac{f(n)}{2n+1} \right)-\sum_{n=0}^\infty \left( \frac{f(n)+1}{2n+2} \right)$$\\
$$=\sum_{n=1}^\infty \left( \frac{f(n)}{2n} \right)+1+\sum_{n=1}^\infty \left( \frac{f(n)+1}{2n+1} \right)
-\sum_{n=1}^\infty \left( \frac{f(n)}{2n+1} \right)-\frac{1}{2}-\sum_{n=1}^\infty \left( \frac{f(n)+1}{2n+2} \right)$$\\
$$=\frac{1}{2}+\sum_{n=1}^\infty \left( \frac{f(n)(2n+2)-(f(n)+1)(2n)}{2n(2n+2)} \right)+\sum_{n=1}^\infty \left( \frac{1}{2n+1} \right)$$\\
$$=\frac{1}{2}+\sum_{n=1}^\infty \left( \frac{2nf(n)+2f(n)-2nf(n)-2n}{2n(2n+2)} \right)+\sum_{n=1}^\infty \left( \frac{1}{2n+1} \right)$$\\
$$=\frac{1}{2}+\sum_{n=1}^\infty \left( \frac{f(n)-n}{2n(n+1)} \right)+\sum_{n=1}^\infty \left( \frac{1}{2n+1} \right)$$\\

\end{document}