\documentclass{article}
\usepackage{hyperref}
\usepackage{times,amssymb,amsmath}
\input{../cs480preamble.tex}

\newcommand{\term}[1]{\textsc{#1}}

\begin{document}

\headerbox{Exploration 1: Or why we should get an A for failing}{Kevin Cherrington, Riley Crebs, Devin Doman, Joe Dzado, Brandon Lyons, Graham Tibbitts}

\section{Algebraic Screw-up(Graham)}
What follows is a semi-faithful recounting of a failed attempt to solve the problem algebraically.
\subsection{Initial assumptions}
These are some assumptions we made in our first attempt to solve the problem with algebra:
1. $f(2n) = f(n)$\\
2. $f(2n-1) = f(n) + 1$\\

(1) turns out to be true, because of the fractal nature of $f(n)$.\\
(2) is false, although assuming it to be true makes for some interesting algebra.\\

\subsection{Flawed reasoning}
$$\sum_{n=1}^\infty \left( \frac{f(n)}{n(n-1)} \right)=\sum_{n=1}^\infty \left( \frac{f(n)}{n} \right)-\sum_{n=1}^\infty \left( \frac{f(n)}{n-1} \right)$$\\
$$\sum_{n=1}^\infty \left( \frac{f(n)}{n} \right)-\sum_{n=1}^\infty \left( \frac{f(n)}{n-1} \right)$$

\section{Riemann Theorem(Riley)}
By using Riemann's theorem we could rearrange the term so the summation would converge the $\ln 4$.  We had hoped that by using Riemann's theorem we could find a one to one mapping between our equation and the equation we had gotten from Riemann's theorem. However, after looking for the one to one mapping for the two equations we found that there was no such mapping.  
\\*
\\* For your view pleasure, Riemann's theorem is as follows \href{http://planetmath.org/encyclopedia/ConditionallyConvergentSeriesOfRealNumbersCanBeRearrangedToConvergeToAnyNumber.html}{{\bf (Online Reference}}):
\\*
\\*
{\bf Theorem (Riemann series theorem)}. Let $(a _{n})$ be a sequence in $\mathbb{R}$ such that $ \sum_{n=1}^{\infty} a _{n} $  converges but $ \sum_{n=1}^{\infty} |a _{n}| = \infty$ , i.e,  $ \sum a_{n} $ is conditionally convergent. Let $\infty \leq \alpha < \beta \leq \infty$  be arbitrary. Then there exists a rearrangement ($a _{n}$) such that $ \displaystyle\liminf_{N}   \sum_{n`=1}^{N} a _{n`} = \alpha$ and $\displaystyle\sum_{n`=1}^{N} a _{n`} = \beta$

\section{Summation Program (Devin)}

\section{Telescoping (Brandon)}

\end{document}