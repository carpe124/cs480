\documentclass{article}
\usepackage{times,amssymb,amsmath}
\input{/home/cs480/latex/cs480preamble.tex}

%%%
%%% To produce a Device Independent (.dvi) file from this one:
%%%
%%%    latex turing.tex
%%%
%%% To view the resulting .dvi file:
%%%
%%%    xdvi turing.dvi
%%%
%%% Alternatively, to convert the DVI file to PDF:
%%%
%%%    dvipdf turing.dvi
%%%

\begin{document}

\handin{Turing Machine Exploration}{PUT YOUR NAME HERE}{PUT THE DATE HERE}

\section{Tractatus Multiplicationis: The Philosophy of Multiplication}
\subsection{What is multiplication?}
\subsubsection{An epistemological answer}
The questions ``what is truth?'' and ``what can be known absolutely?'' are interesting ones, philosophically. Similar questions lead Renee Descartes to conduct his famous thought experiment, resulting in the conclusion ``Cogito ergo sum''. To reach that important conclusion, Descartes chose to systematically set aside most of his other beliefs, because they could be doubted. When he finally found one belief that was undoubtable, he was able to build up an entire belief system around it. The fact of his existence, then, became one of the pillars of his belief.\\
While it may be amusing to consider that our entire existence might be nothing but a grand trick foisted upon us by a malevalent demon, most of us still act as if the world is exactly as it appears. True sceptics may doubt the existence of physical bodies, but they eat, nevertheless. They may question the actual existence of a cliff in their path, but they still keep a safe distance, regardless.\\
There are certain mathematical ideas that are harder to doubt. For instance, even though we may not be able to explain \emph{why} $1 + 1 = 2$, we still agree that it is true. Perhaps that \emph{is} the answer: $1 + 1 = 2$ \emph{because} we agree on it. Really, though, we're only agreeing on the meaning of the symbols. It seems intuitively obvious that when you take $1$ and combine it with another $1$ you'll have $2$ $1$s. Addition in unary seems much easier to visualize and define.\\
Perhaps we can use the same type of trick for defining multiplication. Indeed, it seems we do use it, at least when learning to multiply. $2 \times 3$, or $11 \times 111$, equals $6$, the number of unit squares that fit into a box with width 2 and length 3.\\
Multiplication in any base is an interesting philosophical and mental exercise. It is interesting that $10 \times 100$ in base 2 has the same value as $2 \times 11$ in base 3, which is the same as $2 \times 4$ in base 10 (or base 8, or 16). Numbers seem to have an identity that is separate from the symbols used to represent them, and independant of base. Of course, we could change our definition of multiplication so that this is not the case, but that would be a different game.

\subsubsection{A practical answer}
Multiplication is a shorthand for repeated addition. The problem $2 \times 3$ given previously asks that 2 be added to itself three times. It can be rewritten as $2 + 2 + 2$, although that is rarely done (otherwise, what good is the shorthand?). Another way to represent the multiplication of $x$ and $y$ is $\sum^y_{n=1} x$, renforcing the fundamental idea of multiplication as repeated addition.\\

\subsection{Why multiply?}
"God said unto them, Be fruitful, and multiply, and replenish the earth" - Genesis 1:28\\
Not only is multiplication interesting philosophically, it is also a very useful process. We are a very pragmatic society, so things that are not as useful become less and less used until they fade from common knowledge. There is little risk of that happening with the multiplication. Methods (and symbols) may vary from region to region, so that we may not understand \emph{how} a particular result was reached. But, we can be very confident that given the same numbers, two competent people will reach the same result, regardless of the method employed. Socrates and Napoleon both would have given an answer of 16 to the question $2 \times 8$, although Socrates might not have understood the concept of negative numbers, or even zero (Actually, there is little evidence that Napoleon understood the concept of negative numbers, either, or he might not have tried an extended invasion of Russia).\\
Beside the fact that the principles of multiplication seem to transend time, space, and culture is how useful multiplication really is. Most of the advances and inventions that we so rely upon would not exist without this simple, profound process. Chemistry, physics, engineering, and even textile manufacturing and cooking would be crippled without it. In short, society, civilization, and all that they entail could barely exist without that most excellent of mathematical gems.\\
The answer to the question posed in this sentence is simple. Why do we multiply? Because we \emph{must}. Because it makes us who we are and holds in it our potential for the future.

\section{Turing Machine Implementation}
\subsection{Multiplication Method}
There are many ways to perform multiplication. The simplest method of multiplication is called long multiplication. Long multiplication is performed by multiplying the multiplicand by each digit of the multiplier and then adding the products by the position of the digit in the multiplier. This method was chosen to be implemented in the Turing machine because it is so simple. It also was the first method that was happened upon when searching for methods of multiplication that work with binary.
\subsection{Multiplication Operation}
Once the method for performing the multiplication was selected it was important to determine how the method worked when applied to a Turing machine. It was easy to decide that the first and second tape would contain the inputs and the third tape would hold the output. It also made sense that because JFLAP starts the inputs from the left, that the left would be the least significant digit. After these were decided, it became easy to step through the actions the Turing machine should make.\\
\begin{tabular}{ | c c c  | c c c | c c c | c c c | c c c | c c c c | c c c | }
\hline
H & 1 & 1 & 1 & H & 1 & 1 & 1 & H & 1 & H & 1 & H & 1 & 1 & H &  & 1 & 1 & H & 1 & 1 \\
H & 0 & 1 & H & 0 & 1 & H & 0 & 1 & H & 0 & 1 & H & 0 & 1 &  & H & 0 & 1 & 0 & H & 1 \\
H & & & 0 & H &  & 0 & 0 & H & 0 & 0 & H & 0 & H & 0 &  & H & 0 & 0 & 0 & H & 0 \\
\hline
\end{tabular}\\
\begin{tabular}{ | c c c | c c c c | c c c c | c c c c | c c c c | c c c c | }
\hline
1 & H & 1 & 1 & 1 & H &  & 1 & H & 1 & & H & 1 & 1 &  & H &  & 1 & 1 &  & H & 1 & 1 \\
0 & H & 1 & 0 & H & 1 &  & 0 & H & 1 & & 0 & H & 1 &  &  & 0 & H & 1 & & 0 & 1 & H \\
0 & 1 & H & 0 & 1 & 1 & H & 0 & 1 & 1 & H & 0 & 1 & H & 1 & 0 & H & 1 & 1 & 0 & 1 & H & 1 \\
\hline
\end{tabular}\\
An arbitrary decision was made to have the second tape incremented when it is.
\subsection{JFLAP}
To create the Turing machine in JFLAP, we started by creating a single state and the transitions for ANDing the first input with the character from the second input and printing it to the third tape. Then, when this worked properly we added a carry state so the result to be printed to the third tape would be added instead of being overwritten. Then it was necessary to add a state for moving the heads back to the left and an accept state.

\section{Bit Shifting}
\paragraph\indent
As we reviewed the solution devised for this exploration we attempted to think of ways to improve the speed of our calculations. It occurred to us that the use of bit shifting might be applied as it is done in the hardware of modern processors. With binary a bit shift will have the following effect: $$n << k = n * 2^{k}$$
This could be applied to our machine by writing zeros at the far right edge of the tape input whenever we desire to shift the bits. The number of times to shift one of the tape inputs could be determined by the length of the other input. This would get you in the ball park of where the solution will eventually be. That is to say shifting the bits would save a lot of work needed by the tape head by quickly moving close to the solution.
\paragraph\indent
From there our turning machine would require a few modifications so as to take the necessary steps to reach the desired answer from wherever multiplying by $2^{k}$ lands us, given the remaining values of the input tapes. Bit shifting can take you far, but it can't take you all the way. We began to wonder if there was a more general way to construct our machine such that it was not restricted to binary inputs. This leads us to our thoughts concerning baseless multiplication.

\section{Baseless Multiplication}
\paragraph\indent
A discussion spawned in our collaboration to create a Turing machine to compute decimal multiplication natively. Of course, we spoke of converting decimal to binary, computing, and converting back to decimal. This, in spite of defeating the idea of native computation in decimal sparked the idea of how said conversion would take place. 
\paragraph \indent We thought of the ways that this conversion is done in hardware with bitshifts, addition, blah blah blah, but this would require a large amount of mathematics which, implementing state by state, would be super complicated. So, we took another route. Converting between bases can be simplified to decrementing the source value in its respective base while incrementing the destination value in its respective base. But wait! Increment and Decrement form $\frac{2}{3}$ of TSL.  Why not perform the entire computation in TSL? Providing a source and destination alphabet, one could create a generic base mathmatical Turing machine.

Now I will describe the Turing machine and its algorithm:\newline
\subsection{Turing Machine Description}
$TSL_{Mult}$ is a 7 tape Turing machine\newline
Tape 1 contains the first operand\newline
Tape 2 is a "scratch" area for a temp value of Tape 1\newline
Tape 3 contains the second operand \newline
Tape 4 is a "scratch" area for a temp value of Tape 3\newline
Tape 5 contains the source alphabet.\newline
Tape 6 contains the destination alphabet (May or may not be the same as the source).\newline
Tape 7 contains the result of the computation.\newline

Now, the algorithm, in pseudocode is as follows.\newline

$while{(operand1)}$ // temporary copy of the operands \newline
\indent\indent$while{(operand2)}$\newline
\indent\indent$\{$\newline
\indent\indent\indent$operand1--; $\newline
\indent\indent\indent$results++;$\newline
\indent\indent$\}$\newline\newline

\subsection{Increment}
The formal, allbeit mechanical, version of incrementation is as follows:
We will define the least significant figure as the right-most character in a string representing a number. We must also assume that we know the magnitude of one character from the others, allbeit by arbitrary assignment. When the ordered set of characters is exhausted, the current place is returned to the "zero" character, the next most significant location is incremented, and incrementing resumes in the least significant place. 

$$Alphabet\;A\;= \{0,1,2\}$$

$$i_0\;\;\;\;\;\; = 1\;0\;1$$
$$i++ = 1\;0\;2$$
$$i++ = 1\;1\;0$$
$$i++ = 1\;1\;1$$

\subsection{Decrement}
As you can imagine, decrementing replaces the current symbol with the next lowest in the pecking order. The issue now is when the least significant place reaches the "zero" spot, scan for the next "nonzero" to the left (per our definition or significance). Replace this character with the next lowest character. Now replace all characters (which are all  "zero") with the greatest symbol in the alphabet. Continue decrementing at the least significant place.

$$Alphabet\;A\;= \{0,1,2,3,4\}$$

$$i_0\;\;\;\;\;\; = 1\;0\;3\;0\;1$$
$$i-- = 1\;0\;3\;0\;0$$
$$i--= 1\;0\;2\;4\;4$$
$$i-- = 1\;0\;2\;3\;3$$

\subsection{Wrap It Up}
\paragraph\indent Not only can this Turing machine natively calculate Binary or Decimal multiplication, it can do any representation to any other representation using the same states and transitions. Ironically, writing code to produce this machine on a modern computer would convert the whole lot to Binary anyway. 

\end{document}

%%% Local Variables: 
%%% mode: latex
%%% compile-command: "latex turing.tex && dvipdf turing.dvi"
%%% TeX-master: t
%%% End: 




